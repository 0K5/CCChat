% vim:ft=tex:
%
\documentclass[10pt,a4paper,landscape]{article}
\usepackage[utf8]{inputenc}
\usepackage[german]{babel}
\usepackage[T1]{fontenc}
\usepackage{amsmath}
\usepackage{amsfonts}
\usepackage{amssymb}
\usepackage{makeidx}
\usepackage{graphicx}
\usepackage{lmodern}
\usepackage{kpfonts}
\usepackage{tabularx}
\usepackage{multirow}
\usepackage{xcolor}
\usepackage{booktabs}
\usepackage{hyperref}
\hypersetup{
    colorlinks = true,
    linkcolor=blue,
    filecolor=magenta,      
    urlcolor=cyan,
}
\usepackage{listings}
\usepackage{color}

\definecolor{dkgreen}{rgb}{0,0.6,0}
\definecolor{gray}{rgb}{0.5,0.5,0.5}
\definecolor{mauve}{rgb}{0.58,0,0.82}

\lstset{frame=tb,
    language=Java,
    aboveskip=3mm,
    belowskip=3mm,
    showstringspaces=false,
    columns=flexible,
    basicstyle={\small\ttfamily},
    numbers=none,
    numberstyle=\tiny\color{gray},
    keywordstyle=\color{blue},
    commentstyle=\color{dkgreen},
    stringstyle=\color{mauve},
    breaklines=true,
    breakatwhitespace=true,
    tabsize=3
}

\usepackage[left=2cm,right=2cm,top=2cm,bottom=2cm]{geometry}

\title{CCChat Documentation}
\date{\today}
\author{Leonid Immel and Oliver Schneider}

\begin{document}
\maketitle
\tableofcontents
\newpage
\section{Up next}
\begin{enumerate}
        \item
                Code documentation
        \item
                moment as date parser 
        \item
                rename helpers to modules
        \item
                database fallback save data back to mlab on reconnect and delete from local mongodb
        \item
                password check for safe password
        \item
                distribute services
        \item
                write a stress test
        \item 
                email authentication with student email
        \item
                media compression and persistency
        \item
                online status
        \item 
                profile picture
\end{enumerate}

\section{Node Modules (Serverside)}
\begin{tabularx}{\textwidth}{ll}
    \textbf{Module} & \textbf{Benefit}               \\
    \toprule
    express         & Web application framework that includes several useful features for web applications. \\ 
    \midrule
    socket.io       & Realtime event-based bidirectional communication. \\
    \midrule
    socket.io-stream & Sending media over socket within stream\\
    \midrule
    winston         & Logger. \\
    \midrule
    body-parser     & Handles http post requests. Extracts body of http to req.body. \\
    \midrule
    serve-favicon   & Handles the requests from browsers for the favicon and caches the favicon. \\
    \midrule
    express-handlebars & Builds semantic templates \\
    \midrule
    mongoose        & For MongoDB modeling. Saves all app data. \\
    \midrule
    redis           & For Redis Connection. Saves the cookies. \\
    \midrule
    session-file-store & Fallback for failed database connections.\\
    \midrule
    mocha           & Testframework \\
    \midrule
    chai            & Helper for testframework \\
    \midrule
    express-brute   & Prevents bruteforce attacks on login \\
    \midrule
    bcrypt          & Secure hashing of passwords in database \\
    \midrule
    express-session & Sessions, so that the user won't have to login all the time\\
    \midrule
    express-socket.io-sessions & Creates a shared session for sockets\\
    \midrule
    fs-extra        & More functionality for fs module (needed to save data to tmp)\\
    \midrule
    moment          & Calculates time from now to certain point. For bruteforce module\\
    \midrule
    popper          & Required by bootstrap 4.0.0 Realtime Cross-Browser automation\\
    \midrule
    winston         & Logger\\
    \bottomrule
\end{tabularx}

\section{Vendor Libraries (Clientside)}
\begin{tabularx}{\textwidth}{ll}
    \textbf{Library}& \textbf{Benefit}       \\
    \toprule
    bootstrap       & Responsive website programming for mobile-first implementation.\\
    \midrule
    jQuery          & Easier handling of javascript \\
    \midrule
    socket.io       & To connect socket from client side\\
    \midrule
    socket.io-stream & To stream media from client side\\
    \midrule
    nw-dialog       & Dialog to upload files from the client\\
    \bottomrule
\end{tabularx}
    
\section{Project Structure}
\begin{itemize}
        \renewcommand{\labelitemi}{$--$}
        \renewcommand{\labelitemii}{$--$}
        \renewcommand{\labelitemiii}{$--$}
        \renewcommand{\labelitemiv}{$--$}
        \item
            app\\
            \begin{itemize}
                    \item
                        public (all static served files)\\
                        \begin{itemize}
                                \item
                                    js\\
                                \item 
                                    css\\
                                \item 
                                    vendor\\
                        \end{itemize}
                    \item
                        services (All services for the app (for example login-Service and chat-Service)\\
                        \begin{itemize}
                                \item
                                    index\\
                                \item
                                    login\\
                                \item 
                                    chat\\
                        \end{itemize}
                    \item
                        helpers (functionality and code shared in the whole app (regex- or renderoperations))\\
                    \item
                        tests (applicationtests)\\
                    \item
                        layouts (Layouts for handlebars)\\
                    \item
                        certs (Certification files for the https)\\
                    \item
                        tmp (all tmp files: logs, modules for handlebars,\dots)\\
            \end{itemize}
\end{itemize}
\renewcommand{\labelitemi}{$\bullet$}
\renewcommand{\labelitemii}{$\cdot$}
\renewcommand{\labelitemiii}{$\diamond$}
\renewcommand{\labelitemiv}{$\ast$}

\section{Code Documentation}
\begin{description}
        \item[modules]
            The functionality of modules is described within a comment on top of the js-file.\\
        \item[functions]
            Functions with complex implementations or unclear names (should never happen!) need a commentation\\
            Simpler functions are not commented to produce more readable code\\
        \item[variables]
            Variables require clear and good naming, never comment a variable unless absolutely necessary.\\ 
            You should think a littel longer and find a better name.\\
\end{description}
\section{SSL}
\subsection{Create certificate}
\begin{lstlisting}
openssl genrsa -out client-key.pem 2048
openssl req -new -key client-key.pem -out client.csr
openssl x509 -req -in client.csr -signkey client-key.pem -out client-cert.pem
\end{lstlisting}
\subsection{Allow Certificate in browser for Dev}
\subsubsection{Chromium / Chrome 64}
Option 1:\\
\begin{enumerate}
        \item
                Go to Settings. Hit the „Advanced“ button on bottom of the page
        \item
                Click the „Manage certificates“ button in the „Privacy and security“ Tab
        \item
                Click the „Authorities“ Tab
        \item
                Click on „Import“
        \item
                Navigate to the cert file within your app source code folder.
        \item
                Click open and allow what you want to allow
\end{enumerate}
Option 2:\\
\begin{enumerate}
        \item
                Open the side explitetly with the url „https://localhost:3000“
        \item
                Click the advanced Button
        \item
                Allow the certificate
        \item 
                Continue with Finish (see a little further down)
        \item
                follow option 2 until point 3.
        \item
                find your certificate in the list of authorities. 
        \item
                click expand on the certificate 
        \item 
                click the 3 dots that appear under the certificate
        \item 
                click „edit“
        \item 
                allow what you want to allow
\end{enumerate}
\subsubsection{Firefox 59}
\begin{enumerate}
        \item
                Open the side explitetly with the url „https://localhost:3000“
        \item
                Click the „Advanced“ Button
        \item 
                Click „Add Exception“
        \item
                Click „Confirm Security Exception“
\end{enumerate}
\section{Logger}
    We use winston for logging.\\
    Logstati:\\
    \begin{description}
            \item[info]
                    Generic info\\
            \item[warn]
                    Can be ignored, there will be an fallback\\
            \item[error]
                    Should not be ignored and should be fixed\\
            \item[debug]
                    For debugging in development\\
    \end{description}
\section{Error Handling}
\section{Useful websites}
\begin{itemize}
        \item
            \href{https://socket.io/get-started/chat/}{Simple chat with socket.io}
        \item
            \href{www.handlebars.com}{Handlebars}
        \item 
            \href{https://www.packtpub.com/books/content/getting-started-react-and-bootstrap}{Combine Bootstrap and ReactJS}
        \item
            \href{http://mongoosejs.com/docs/guide.html}{Mongoose}
        \item
            \href{https://github.com/phutchins/pipo}{Example of secure chat in production}
        \item
            \href{https://support.securly.com/hc/en-us/articles/206081828-How-to-manually-install-the-Securly-SSL-certificate-in-Chrome}{Add a SSL-Cert in Chrome}
\end{itemize}

\end{document}
