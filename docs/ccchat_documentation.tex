% vim:ft=tex:
%
\documentclass[10pt,a4paper,landscape]{article}
\usepackage[utf8]{inputenc}
\usepackage[german]{babel}
\usepackage[T1]{fontenc}
\usepackage{amsmath}
\usepackage{amsfonts}
\usepackage{amssymb}
\usepackage{makeidx}
\usepackage{graphicx}
\usepackage{lmodern}
\usepackage{kpfonts}
\usepackage{tabularx}
\usepackage{multirow}
\usepackage{xcolor}
\usepackage{booktabs}
\usepackage{hyperref}
\hypersetup{
    colorlinks = true,
    linkcolor=blue,
    filecolor=magenta,      
    urlcolor=cyan,
}
\usepackage[left=2cm,right=2cm,top=2cm,bottom=2cm]{geometry}

\title{CCChat Documentation}
\date{\today}
\author{Leonid Immel and Oliver Schneider}

\begin{document}
\maketitle
\tableofcontents

\section{Node Modules (Serverside)}
\begin{tabularx}{\textwidth}{ll}
    \textbf{Module} & \textbf{Benefit}               \\
    \toprule
    express         & Web application framework that includes several useful features for web applications. \\ 
    \midrule
    passport        & Authentication middleware for session management. \\
    \midrule
    socket.io       & Realtime event-based bidirectional communication. \\
    \midrule
    winston         & Logger. \\
    \midrule
    body-parser     & Handles http post requests. Extracts body of http to req.body. \\
    \midrule
    serve-favicon   & Handles the requests from browsers for the favicon and caches the favicon. \\
    \midrule
    handlebars      & Builds semantic templates \\
    \midrule
    mongoose        & For MongoDB modeling \\
    \midrule
    mocha           & Testframework \\
    \bottomrule
\end{tabularx}

\section{Vendor Libraries (Clientside)}
\begin{tabularx}{\textwidth}{ll}
    \textbf{Library}& \textbf{Benefit}       \\
    \toprule
    bootstrap       & Responsive website programming for mobile-first implementation.\\
    \midrule
    reactjs         & Helps to create user interfaces \\
    \bottomrule
\end{tabularx}
    
\section{Project Structure}
\begin{itemize}
        \renewcommand{\labelitemi}{$--$}
        \renewcommand{\labelitemii}{$--$}
        \renewcommand{\labelitemiii}{$--$}
        \renewcommand{\labelitemiv}{$--$}
        \item
            app\\
            \begin{itemize}
                    \item
                        public (all static served files)\\
                        \begin{itemize}
                                \item
                                    js\\
                                \item 
                                    css\\
                                \item 
                                    html\\
                                \item
                                    libs\\
                        \end{itemize}
                    \item
                        views (handlebar views)\\
                    \item
                        models (handles data, logic and storage of the app)\\
                    \item
                        controllers (defines the routes within the app)\\
                    \item
                        middleware (processes incoming requests before they are passed to the routes (loginauthentication))\\
                    \item
                        helpers (functionality and code shared in the whole app (regex- or renderoperations))\\
                    \item
                        tests (all tests for the application)\\
            \end{itemize}
\end{itemize}
\renewcommand{\labelitemi}{$\bullet$}
\renewcommand{\labelitemii}{$\cdot$}
\renewcommand{\labelitemiii}{$\diamond$}
\renewcommand{\labelitemiv}{$\ast$}

\section{Rules in documentation and formatting}
\subsection{Documentation}
\begin{description}
        \item[modules]
            The functionality of modules is described within a comment on top of the js-file.\\
        \item[functions]
            Functions with complex implementations or unclear names (should never happen!) need a commentation\\
            Simpler functions are not commented to produce more readable code\\
        \item[variables]
            Variables require clear and good naming, never comment a variable unless absolutely necessary.\\ 
            You should think a littel longer and find a better name.\\
\end{description}
\section{Logger}
\section{Error Handling}
\section{Useful websites}
\begin{itemize}
        \item
            \href{https://socket.io/get-started/chat/}{Simple chat with socket.io}
        \item
            \href{www.handlebars.com}{Handlebars}
        \item 
            \href{https://www.packtpub.com/books/content/getting-started-react-and-bootstrap}{Combine Bootstrap and ReactJS}
        \item
            \href{http://mongoosejs.com/docs/guide.html}{Mongoose}
        \item
            \href{https://github.com/phutchins/pipo}{Example of secure chat in production}
\end{itemize}
\end{document}
